% Use only LaTeX2e, calling the article.cls class and 12-point type.

\documentclass[11pt]{article}

\usepackage[round,semicolon]{natbib}
\usepackage{etoolbox}
\AtBeginEnvironment{quote}{\singlespacing\tiny}
% Use times if you have the font installed; otherwise, comment out the
% following line.

\usepackage{times}
\usepackage{amssymb}
\usepackage{amsmath}

% for adjustwidth
\usepackage{changepage}

% The following parameters seem to provide a reasonable page setup.

\topmargin 0.0cm
\oddsidemargin 1cm
\textwidth 15cm 
\textheight 21cm
\footskip 1.0cm

\usepackage{newfloat}
\usepackage{amsmath}
\usepackage[labelfont=bf]{caption}
\usepackage{nameref}
\usepackage{rotating}
\usepackage{color}
\usepackage{float}

\usepackage[dvipsnames]{xcolor}

\newcommand\skhcomment[1]{{\color{violet}[#1]}}
\newcommand\wdcomment[1]{{\color{orange}[#1]}}
\newcommand\amcomment[1]{{\color{teal}[#1]}}


\usepackage{hyperref}
\hypersetup{colorlinks,citecolor=blue,linkcolor=blue,urlcolor=blue}
\hypersetup{colorlinks,citecolor=blue,linkcolor=blue,urlcolor=blue}

\title{Models} 

\author{Will DeWitt, Sarah Hilton, Andy Magee}

% Include the date command, but leave its argument blank.
\date{}
\usepackage{setspace}
\onehalfspacing

%%% Beginning of the Document
\begin{document} 
\maketitle 

\section*{Goal and Housekeeping} 

Andy re-implemented the model from \cite{nasrallah2013phylogenetic}, which models pair-wise epistasis due to RNA secondary structure. 
Below is the model as Andy implemented it in \texttt{RevBayes}. 
\skhcomment{Comments from Sarah look like this.}
\wdcomment{Comments from Will look like this.}
\amcomment{Comments from Andy look like this.}

\section*{RNA epistasis model}

Here is the model from \cite{nasrallah2013phylogenetic}. 
\skhcomment{I am just copying over what is found in the paper. We should change this around in an order we like and with notation we like.}

$\boldsymbol{Q}$ is the instantaneous rate matrix describing changes from doublet $\boldsymbol{x}$ to doublet $\boldsymbol{y}$.  
For $\boldsymbol{x} = (x_1, x_2)$, $x_1$ is the 5' nucleotide and $x_2$ is the 3' nucleotide. 
\begin{equation}
\label{eq:Q}
\boldsymbol{Q} = 
\begin{cases}
   \xi \pi_y S_{x_1, y_1} & \mbox{if single substitution at the 5' site,} \\
   \xi \pi_y S_{x_2, y_2} & \mbox{if single substitution at the 3' site,} \\
   \xi \pi_y S_{x_1, y_1} S_{x_2, y_2} d & \mbox{if double substitution where $x$ and $y$ $\in W$,} \\
   0 & \mbox{if any other double substitution,} \\
   - \sum_{y \ne x} q_{xy}& \mbox{if $x$ = $y$} \\
   \end{cases}
\end{equation}
$\boldsymbol{S}$ is the GTR exchangeability matrix \citep{tavare1986some},  
$W = {AT, CG, GC, TA}$ is the set of Watson-Crick pairs, 
$\boldsymbol{\pi} = (\pi_{AA}, \pi_{AC}, ..., \pi_{TT})$ are the stationary state frequencies of the 16 possible doublet states, 
$d$ is the relative rate of double to single mutations between doublets, 
and $\xi$ is the rate-scaling factor. 

\subsection*{Points to clarify}

\subsubsection*{How do we interpret $d$?}

$d$ is the relative rate of double to single mutations between doublets or the "strength" of epistatic interactions. 

Here is a copy of table 1 from \cite{nasrallah2013phylogenetic}: 

\begin{tabular}{ |p{1cm}|p{5cm}|p{5cm}|  }
 \hline
& $\pi_y = \pi_{y_1}\pi_{y_2}$&$\pi_y \ne \pi_{y_1}\pi_{y_2}$\\
 \hline
$d=0$  & Independent and nonepistatic & Model inadequacy \\
\hline
 $d > 0$  & Dependent but nonepistatic & Dependent and epistatic\\
 \hline
\end{tabular}

\skhcomment{We talked about this last time but I want to make sure I totally understand. When we are simulating with $d=0$, we are actually in the ``model inadequacy" quadrant?}

\subsubsection*{How do we normalize $\boldsymbol{Q}$ ($\xi$)?}

\skhcomment{From slack: 0.5 * Pr(single) + Pr(double) }

\subsubsection*{Given the fit parameter values for a GTR model, how do we simulate under this model?}

\skhcomment{From github, ``
This is the tree inferred under GTR+GAMMA using RAxML version 8.2.12. Inferred model parameters

alpha shape parameter = 0.440894
relative exchange rates (ac ag at cg ct gt) = 1.882161 7.009179 0.914813 0.495852 7.666181 1.000000
base frequencies = 0.340152 0.190828 0.225045 0.243974"}

\clearpage 
\bibliographystyle{mbe}
{\small
\bibliography{references.bib}
}


\end{document}
